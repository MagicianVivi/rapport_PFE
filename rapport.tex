\documentclass[a4paper,12pt]{article}
\usepackage[utf8]{inputenc}
\usepackage{graphicx}
\usepackage{fancyhdr}
\usepackage[pdftex=true, hyperindex=true, colorlinks=true, linkcolor=blue,
            citecolor=cyan]{hyperref}
\usepackage[francais]{babel}

\begin{document}

\begin{titlepage}

\addtolength{\oddsidemargin}{-0.15in}
\addtolength{\textwidth}{0.5in}
\addtolength{\topmargin}{-.375in}
\addtolength{\textheight}{0.75in}

\begin{center}

\begin{tabular*}{6in}{l@{\extracolsep{\fill}}r}{Vivien Alger}&{À
    l'attention de Florent Devin,}\\
    {Éric Hostalery}&{Yannick Le Nir,}\\
    {Cédric Salvador}&{et Rémi Vernay}\\
    {Mathieu Vacher}
\end{tabular*}
\vspace*{\fill}

\textsc{\LARGE Étude de la faisabilité
d'un cluster de smartphones Android}\\
\vspace*{\fill}

\today
\end{center}

\end{titlepage}

\tableofcontents
\newpage

% Defining page style for the rest of the document
\pagestyle{fancy}
\fancyhf{}
\fancyhead[L]{Introduction}
\fancyfoot[C]{Vivien Alger, Éric Hostalery, Cédric Salvador et Mathieu Vacher}
\fancyhead[R]{\thepage}
\renewcommand{\footrulewidth}{1pt}
\renewcommand{\headrulewidth}{1pt}
  
\section*{Introduction}
Dans le cadre de notre projet de fin d’études nous avons choisi de nous 
attarder sur le comportement d’un cluster de smartphones. Plus précisément
sur la plateforme Android, vu que c’est actuellement la plus “ouverte”, 
accessible et répandue.\\
En effet, nous avons constaté que les terminaux mobiles deviennent de plus en 
plus puissants depuis quelques années, au point que l’on peut désormais faire 
tourner des systèmes d’exploitation comme GNU/Linux dessus. Partant de ce 
constat, nous avons alors décidé d’explorer les spécificités du calcul réparti
sur de telles machines, et de nous concentrer sur les aspects inhérents à ces 
plateformes que sont la consommation de la batterie et les déconnexions/pertes
 de réseau intempestives.\\
Ainsi nous avons cherché à mettre en place une gestion dynamique de 
l’architecture de calcul pour gé́rer la consommation de batterie et la tolé́rance
aux pannes lié́es au ré́seau mobile.\\
Pour illustrer cette problématique, nous avons tenté de porter
Akka sur la plateforme Android en le couplant avec le service google cloud 
messaging pour faire communiquer les différent téléphones.
\newpage

\fancyhead[L]{\leftmark}

\section{Problématique}
\section{Pistes architecturales}
\subsection{Architecture type Hadoop}
\subsection{Architecture type Akka/Gridgain}
\subsection{Architecture avec uniquement des SMS et base de données embarquée}
\section{Application d'une architecture: tentative de portage d'Akka}
\subsection{Scala et Akka sur Android}
\subsection{Utilisation de Google Cloud Messaging}
\subsection{Intégration avec des SMS}
\subsection{Alternative avec le projet Akka-Mobile}
\newpage

\fancyhead[L]{Conclusion}

\section*{Conclusion}
\end{document}
